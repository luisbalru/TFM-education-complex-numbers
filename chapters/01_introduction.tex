\documentclass[../main.tex]{memoir}

\begin{document}

\chapter{Introducción}

Alrededor del año 1575, Rafael Bombelli, brillante algebrista italiano, decía lo siguiente sobre las soluciones complejas de una ecuación cúbica: \\

\textit{``Al principio, me parecía que estaban basadas más en sofismos que en argumentos reales y ciertos, pero continué buscando hasta que encontré la prueba.''} \\

Esta es la historia de un concepto que intentó ser negado y, por mucho que las más brillantes mentes matemáticas lo intentaron, emergió para convertirse en una teoría tan \textit{real} y necesaria, que hoy da soporte a los mayores avances en ciencia y tecnología conocidos, como la física cuántica, la química cuántica, los circuitos electrónicos, electromagnetismo o la  dinámica de fluidos. \\

Desde el punto de vista matemático, una de las grandes implicaciones que trajo consigo la variable compleja es el Teorema Fundamental del Álgebra. Ya en 1637, René Descartes ponía de manifiesto que la teoría de los números reales era incompleta: \\

\textit{``Para cualquier ecuación (polinómica) uno puede imaginar tantas raíces (como le indique su grado) pero en múltiples ocasiones no existe tal cantidad que corresponda a lo que imaginamos.''} \\

Este sería el nacimiento del término \textit{imaginario}. A pesar de que multitud de matemáticos fueron introduciendo paulatinamente conceptos, términos y metodologías para trabajar con números complejos, aún Cauchy en 1825 decía \\

\textit{``Repudiamos completamente el símbolo $\sqrt{-1}$, abandonándolo sin arrepentimiento porque no sabemos lo que este supuesto símbolo significa ni qué significado darle.''} \\



En mi opinión, los contenidos referentes a números complejos son especialmente importantes por varios motivos. En primer lugar, porque es la primera (y última) vez que el alumnado se enfrentará a ellos en su etapa de Educación Secundaria Obligatoria y Bachillerato. Además, desde que empiezan a estudiar ecuaciones de segundo grado, se encuentran con que algunas de ellas, aparentemente sencillas, no tienen solución. Con este bloque se cierra ese círculo, resolviéndose incertidumbres arrastradas durante años. Además, lejos de enseñar a los alumnos esta nueva forma de ver las ecuaciones, los números y su representación desde un prisma meramente mecánico, pienso que es necesario que el alumnado conozca de primera mano la historia de los números complejos para que vean que los grandes pensadores de nuestra disciplina intentaron continuar sus investigaciones sin tener ninguna fe en la validez de estos números, por ser disruptivos frente a la tradición de siglos, pero que aún así fueron capaces de cambiar sus convicciones para dar cabida a una teoría que revolucionaría la vida de todos los seres humanos del planeta a partir del siglo XX hasta nuestros días.\\


Depositaremos todo nuestro rigor en las siguientes secciones para diseñar una estrategia de aprendizaje eficiente, que ligue la visión geométrica y algebraica de los números complejos, para conseguir que los alumnos y alumnas acaben el curso con una percepción amplia y rica de todos los conjuntos numéricos, incluyendo el cuerpo de los números complejos, y todas sus aplicaciones.

\end{document}

%%% Local Variables:
%%% mode: latex
%%% TeX-master: "../main"
%%% End:
