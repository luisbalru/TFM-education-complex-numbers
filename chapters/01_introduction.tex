\documentclass[../main.tex]{book}

\begin{document}

\chapter{Introducción}

En las últimas décadas se ha experimentado un crecimiento poblacional
sin precedentes alrededor de todo el mundo, con el consecuente aumento
de las aglomeraciones de personas, las cuales llegan a involucrar a
miles de individuos. Además, las tasas de criminalidad y terrorismo se
han disparado de forma similar. La combinación de estos hechos ha
provocado que la videovigilancia masiva se convierta en una
herramienta prioritaria. El número de cámaras de vigilancia instaladas
en el mundo, tanto dentro del ámbito público como en el privado, se ha
multiplicado en los últimos años. El desarrollo tecnológico, además,
está produciendo una importante mejora en la calidad de los vídeos que
se recopilan, a cambio de un aumento importante en el volumen de
información almacenada. Aparece, por tanto, la necesidad de procesar
una gran cantidad de información en forma de archivos de vídeo.\\

Históricamente, dicho procesamiento se ha realizado de forma manual,
consumiendo una gran cantidad de recursos humanos. Hoy en día, la
velocidad a la que se genera dicha información, y el volumen tan
abismal que se genera diariamente, hace casi imposible la gestión
manual de esta información de forma exhaustiva y adecuada. Además,
esta información debe procesarse en tiempo real en la medida de lo
posible, ya que la respuesta rápida en situaciones de emergencia es
crucial para reducir los efectos de una posible catástrofe. Esto ha
hecho que los métodos clásicos de supervisión humana queden
paulatinamente obsoletos, y aparezca la necesidad de automatizar el
proceso. En este contexto aparece el concepto de la videovigilancia
automática.\\

La videovigilancia automática es una rama de investigación cuyo
objetivo es el análisis de múltiples fuentes de vídeo en tiempo real,
para la extracción automática de información relevante relacionada con
el comportamiento de los individuos \cite{ma2009intelligent}. Esta
área de investigación aúna dos grandes campos de trabajo dentro del
aprendizaje automático; la visión por computador y el análisis de
series temporales. Dado que el tipo de dato más común dentro de este
contexto son las secuencias de vídeo, por un lado se ha de extraer
información de cada uno de los fotogramas, y por otro información
temporal derivada de la secuencia de dichos fotogramas.\\

El auge del aprendizaje profundo, además, ha supuesto un avance muy
importante en el desarrollo de modelos en este contexto. Según
\cite{zitouni2016advances}, existen cuatro áreas principales en el
contexto de la videovigilancia automática:

\begin{enumerate}
\item Detección y seguimiento de individuos
\item Recuento y estimación de densidad de individuos
\item Análisis y clasificación de comportamientos
\item Detección de comportamientos anómalos
\end{enumerate}

Las tareas 1 y 2 han sido ampliamente estudiadas y los modelos de
aprendizaje clásico son suficientes para la obtención de resultados de
calidad. No obstante, en las tareas 3 y 4 los resultados eran muy
limitados. La aparición de los modelos de aprendizaje profundo y el
aumento de la capacidad de cálculo ha supuesto un avance importante
para todas las áreas. En los dos primeros casos, ha permitido que la
densidad de individuos presentes en la imagen sea más alta antes de
que se produzca una pérdida de rendimiento, y en los dos últimos casos
ha provocado una mejoría muy notable, ya que la complejidad de la
información que estos modelos son capaces de extraer es notablemente
más alta que la extraída por los modelos clásicos.\\

Esta mejora tan significativa ha provocado que en los últimos años
aparezcan una gran cantidad de trabajos que resuelven alguno de los
problemas relacionados con la videovigilancia automática aplicando
modelos de redes neuronales. No obstante, estos trabajos aparecen
dispersos, y es difícil establecer una comparativa sobre ellos. Esta
problemática es especialmente relevante cuando se tratan de
desarrollar nuevos modelos, ya que es difícil recopilar el
conocimiento previo sobre la temática. Esta dispersión radica en
varios factores:\\

\begin{enumerate}
\item No existe un consenso claro sobre las tareas que deben abordarse
  dentro de esta área de investigación.
\item No hay una taxonomía clara para organizar los distintos trabajos
  previamente desarrollados.
\item No existe una recopilación de trabajos que afronten esta
  problemática desde la perspectiva del aprendizaje profundo.
\end{enumerate}

En particular, nuestra propuesta de trabajo se engloba dentro del
proyecto de investigación \textit{AI\_MARS-DeepLABD: Artificial
  Intelligence system for Monitoring, Alert and Response for Security
  in events. Deep Learning for Abnomal Behavior Detection}, el cual
pretende diseñar e implementar sistemas de aprendizaje profundo para
la detección de comportamientos anómalos, y por tanto, enfocaremos el
trabajo en esa dirección. Esto hace que nos centremos especialmente en
la última de las tareas. Es la más novedosa de las áreas listadas
anteriormente, y esto hace que exista una especial incertidumbre
alrededor de la misma. En particular, resulta muy difícil establecer
una organización clara para los trabajos que afrontan esta
problemática, porque el concepto de comportamiento anómalo puede
incluir definiciones muy diversas. Por ejemplo, podemos considerar
como anómalo la presencia de una persona en un área restringida, una
multitud corriendo despavorida, o un pequeño grupo de personas que
inicia una pelea por la calle. Claramente, la fuente de la anomalía en
los tres casos es completamente distinta, y difícilmente comparable.
Esto provoca que la comparativa entre modelos sea compleja. Además, el
número de conjuntos de datos públicos que permitan establecer
comparaciones entre los modelos es relativamente escaso, y un gran
número de trabajos utilizan sus propios conjuntos de datos diseñados
específicamente para el problema que tratan de resolver.\\

\section{Objetivos del trabajo}

Dado el contexto previo, los objetivos de este trabajo tratan de
cubrir un estudio profundo del área de la videovigilancia automática,
en particular centrado en la detección de comportamientos anómalos en
vídeo. Los objetivos concretos que se han planteado para el trabajo
son los siguientes:

\begin{enumerate}
\item Proponer una taxonomía para la organización de los trabajos que
  afrontan el problema del análisis de multitudes en videovigilancia.
\item Revisar detalladamente los trabajos propuestos dentro del área
  de la detección de comportamientos anómalos utilizando aprendizaje
  profundo.
\item Estudiar la extracción de características en vídeo utilizando
  modelos de aprendizaje profundo. En concreto, se estudia un modelo
  del estado del arte en la detección de anomalías en videovigilancia,
  y se propone una mejora basada en un extractor de características
  profundo de mayor potencia.
\end{enumerate}

El resto del trabajo se estructura de la siguiente manera. En el
capítulo \ref{sec:state-of-the-art} se expone el estudio teórico del
trabajo. En él, se propone una taxonomía en etapas que permite agrupar
los trabajos hasta el momento dentro de cuatro etapas, en las que cada
una recae en los resultados de las anteriores. Además, se estudian los
principales conjuntos de datos disponibles públicamente y las métricas
que se utilizan para evaluar la calidad de los modelos dentro de esta
área de conocimiento. Finalmente, se resumen los principales trabajos
que utilizan aprendizaje profundo para resolver el problema de la
detección de anomalías en multitudes, estableciendo una división de los
mismos en función de la taxonomía previa.\\

En el capítulo \ref{sec:model-analysis} se llevará a cabo el análisis
del uso de características espacio-temporales para la detección de
acciones anómalas. Concretamente, tomaremos el trabajo propuesto en
\cite{sultani2018real} y trataremos de mejorar sus resultados
empleando un extractor de características más potente. En dicho
trabajo, proponen una red neuronal convolucional en tres dimensiones
como extractor de características en vídeo. A pesar de ser un modelo
relativamente bueno para capturar características tanto temporales
como espaciales, creemos que las características temporales se ven
infrarrepresentadas. Nuestra hipótesis expone que el uso de un
extractor de características más potente, que combine la capacidad de
trabajar con imágenes de las redes convolucionales 2D con la capacidad
de analizar series temporales de las redes recurrentes, obtendrá
mejores resultados en el problema tratado. En este capítulo se realiza
un estudio teórico del modelo original y se realiza nuestra propuesta
de mejora.

A continuación, en el capítulo \ref{sec:experiments-and-results} se
detalla la experimentación realizada y los resultados
obtenidos. Finalmente, en \ref{sec:conclusions-future-work} se exponen
las conclusiones derivadas del estudio y posibles estudios futuros.

\end{document}

%%% Local Variables:
%%% mode: latex
%%% TeX-master: "../main"
%%% End:
