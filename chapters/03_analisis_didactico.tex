\documentclass[../main.tex]{memoir}

\begin{document}

\chapter{Análisis Didáctico}
\label{sec:analisis-didactico}

El conocimiento de los conceptos matemáticos por parte del profesor no garantizan la correcta enseñanza de los mismos. Es necesario un análisis profundo donde, por una parte, se analicen los contenidos a enseñar y su organización; por otra, la situación del alumnado. La unión de estas dos circunstancias guiará al profesor en la elección y diseño de actividades propicias y de un buen método de evaluación. En definitiva, es necesario llevar  a cabo un análisis didáctico. Segun \cite{rico2016}, el análisis didáctico es \textit{``el sistema y método de trabajo para el profesor de matemáticas, entre cuyas funciones destaca la reflexión sobre la estructura del currículo de matemáticas, junto con un dominio técnico para planificar e implementar unidades matemáticas escolares''.} \\

En esta sección, abordamos el Análisis de Contenidos, que estudia el sentido de los contenidos y su organización; el Análisis Cognitivo, que se centra en el aprendizaje de un tema matemático por parte de los estudiantes; el Análisis de Instrucción, en el que se deciden y diseñan ejercicios idóneos para el aprendizaje de los contenidos; y el Análisis Evaluativo, como pieza clave para la aplicación de un adecuado sistema de evaluación de lo aprendido.

\section{Análisis de Contenidos}

¿Qué es conocer un concepto matemático? Según \cite{rico2016}, \textit{``conocer su definición, representarlo, mostrar sus operaciones, relaciones y propiedades y los modos de uso, interpretación y aplicación a la resolución de problemas''.} Podemos llevar a cabo este conocimiento por dos vías. La primera, en la que aunamos la estructura conceptual y sus sistemas de representación, generan un significado instrumental del concepto, logrando su dominio por medio de un aprendizaje basado en la memorización de hechos, destrezas y propiedades. Por otra parte, se le puede dar un enfoque funcional, basado en que los conceptos y procedimientos asociados tienen una función y se usan con determinados propósitos. En definitiva, las diversas formas de entender, expresar y usar un concepto constituyen su significado. Por contenido matemático escolarse se entiende un conjunto de conceptos, procedimientos, estructuras y actitudes que los responsables del currículo escogen y organizan, que los profesores comunican y enseñan, para que los escolares aprendan acerca de un tópico matemático escolar determinado y lo utilicen. El análisis del contenido matemático escolar consiste en establecer con detalle sus descriptores y componentes particulares según cada una de las categorías cognitivas de contenido,establecidas por ámbitos y niveles. (\cite{rico2016}). \\

Este análisis se organiza en torno a tres aspectos:

\begin{itemize}
	\item Estructura conceptual. Se relacionan conceptos, procedimientos y actitudes implicados en el contenido objeto de estudio.
	\item Sistemas de Representación. Distintos modos de representación de un concepto matemático de acuerdo a su estructura o sus propiedades.
	\item Sentido. El análisis de contenido se ocupa de estudiar el sentido de un concepto matemático escolar analizando sus términos, contextos, fenómenos y situaciones en lo que se aplica.
\end{itemize}


\subsection{Estructura conceptual}

Según \cite{rico2016}, llamamos \textit{``estructura conceptual al primero de los sistemas de categorías u organizadores que identifica los aspectos formales y los cognitivos con que se caracterizan y describen los contenidos cuyo estudio se considera''}. Como ya comentamos antes, la estructura conceptual se divide en conceptos, procedimientos y actitudes. \cite{rico1997} indica que los conceptos son la sustancia general del conocimiento, aquello que pensamos, mientras que los procedimientos aglutinan los modos actuación y ejecución de tareas matemáticas. Finalmente, el campo actitudinal incluye los aspectos afectivos, éticos y normativos de la disciplina (\cite{rico2016}). Recuperando los contenidos de nuestra unidad didáctica (Real Decreto 1105/2014)

\begin{itemize}
	\item Números complejos
	\item Forma binómica y polar
	\item Representaciones gráficas
	\item Operaciones elementales
	\item Fórmula de Moivre.
\end{itemize}

\begin{table}[H]
	\centering
	\begin{tabular}{lcccccc}
		\toprule
		\hspace{2.5cm}Contenido conceptual \\
		\midrule
		- Parte real, parte imaginaria. Unidad imaginaria \\
		- Forma binómica \\
		- Afijo, módulo y argumento de un número complejo \\
		- Conjugado de un número complejo \\
		- Eje real y Eje Imaginario \\
		- Forma polar \\
		- Forma trigonométrica\\
		\bottomrule
	\end{tabular}
	\caption{Conceptos}
	\label{tab:conceptos}
\end{table}


\begin{table}[H]
	\centering
	\begin{tabular}{lcccccccccccc}
		\toprule
		\hspace{4cm}Contenido procedimental \\
		\midrule
		- Identificar la parte real y parte imaginaria de un número complejo \\
		- Calcular el afijo, módulo y argumento de un número complejo \\
		- Calcular el conjugado de un número complejo \\
		- Representar números complejos en su forma binómica y polar\\
		\hspace{0.2cm} en el plano complejo (Eje Real y Eje Imaginario) \\
		- Pasar de forma binómica a forma polar \\
		- Pasar de forma binómica y polar a forma trigonométrica\\
		- Operaciones con números complejos en forma binómica: sumar, \\ \hspace{0.2cm}multiplicar y dividir números complejos. Producto por escalar y potencia. \\
		- Operaciones con números complejos en forma polar: Producto, \\
		\hspace{0.2cm}división, raíz $n$-ésima y potencia. Uso de la fórmula de Moivre.\\
		\bottomrule
	\end{tabular}
	\caption{Procedimientos}
	\label{tab:procedimientos}
\end{table}

\begin{table}[H]
	\centering
	\begin{tabular}{lccccc}
		\toprule
		\hspace{4cm}Contenido actitudinal \\
		\midrule
		- Valoración del cuerpo de los números complejos como el conjunto \\ \hspace{0.2cm} numérico que engloba a los conocidos anteriormente ($\mathbb{N}, \mathbb{Z}, \mathbb{Q}, \mathbb{R}$). \\
		- Valoración de la importancia de los números complejos en ciencia y \\ \hspace{0.2cm} tecnología como en la física cuántica, circuitos electrónicos, \\ \hspace{0.2cm} electromagnetismo o dinámica de fluidos. \\
		\bottomrule
	\end{tabular}
	\caption{Actitudes}
	\label{tab:actitudes}
\end{table}


A su vez, cada concepto relacionada y organiza \textbf{hechos}. Se pueden entender como unidades de información que constituyen un primer nivel básico para analizar el ámbito conceptual de un contenido matemático. El estudio de los hechos se hace mediante cuatro categorías: términos, notaciones, convenios y resultados. Los hechos se relacionan y organizan para dar lugar a \textbf{conceptos} y \textbf{estructuras} (\cite{rico2016}). \\



Concretamente,

\begin{table}[H]
	\centering
	\begin{tabular}{llccccccccccccccccccccc}
		\toprule
		\hspace{5.5cm}\textbf{Hechos} \\
		\midrule
		Términos & Notaciones \\
		\midrule
		- Número complejo & - Forma binómica: $z = a +bi$, $\mathbb{C}$ \\
		& - $i$ unidad imaginaria \\
		- Parte real & - Re $z$ \\
		- Parte imaginaria & - Im $z$ \\
		- Afijo & - $P(a,b)$ punto del plano complejo\\
		- Módulo & - $|z|$ \\
		- Argumento & - $Arg z$ \\
		- Eje real, eje imaginario & \\
		 & - Conjugado: $\bar{z}$ \\
		 & - Forma polar: $z = r_\alpha$ \\
		\midrule
		Convenios & Resultados \\
		\midrule
		- El número complejo $z = a +bi$  & - Todo número complejo $z = a +bi$ \\
		se lee ``a más b i'', eludiéndose & se define como un punto $P(a,b)$  \\
		el signo de multiplicación. & del plano complejo  \\
		- Si $a=0, z$ se dice imaginario puro & - El eje real es el eje de abscisa y el eje \\
		&  imaginario es el eje de ordenadas. \\
		& Los números reales son números  \\
		& complejos con parte imaginaria  \\
		& nula ($b=0$) \\
		\bottomrule
	\end{tabular}
	\caption{Contenido conceptual: Hechos. Términos, notaciones, convenios y resultados}
	\label{tab:terminos-notaciones}
\end{table}


\begin{table}[H]
	\centering
	\begin{tabular}{lccccc}
		\toprule
		\hspace{2cm}Conceptos \\
		\midrule
		- Forma binómica \\
		- Forma polar \\
		- Forma trigonométrica\\
		- Raíz compleja de un polinomio \\
		\bottomrule
	\end{tabular}
	\caption{Contenido conceptual: Conceptos}
	\label{tab:conceptos2}
\end{table}

\begin{table}[H]
	\centering
	\begin{tabular}{lcc}
		\toprule
		\hspace{2.5cm}Estructuras \\
		\midrule
		 - ($\mathbb{C}$,+,·): Cuerpo de los números complejos \\
		\bottomrule
	\end{tabular}
	\caption{Contenido conceptual: Estructuras}
	\label{tab:estructuras}
\end{table}



Por otra parte, en el campo procedimental se diferencian, igualmente, tres niveles, según la complejidad del contenido. Son las destrezas, que procesan hechos; los razonamientos, que procesan conceptos; y las estrategias, que procesan estructuras (\cite{rico2016}). \\

\begin{table}[H]
	\centering
	\begin{tabular}{lcccccc}
		\toprule
		\hspace{5.5cm}Destrezas \\
		\midrule
		- Identificar la parte real y parte imaginaria de un número complejo \\
		- Calcular el afijo, módulo y argumento de un número complejo \\
		- Calcular el conjugado de un número complejo \\
		- Representar números complejos en su forma binómica y polar\\
		\hspace{0.2cm} en el plano complejo (Eje Real y Eje Imaginario) \\
		\bottomrule
	\end{tabular}
	\caption{Contenido procedimental: Destrezas}
	\label{tab:destrezas}
\end{table}


\begin{table}[H]
	\centering
	\begin{tabular}{lcccccc}
		\toprule
		\hspace{5.5cm}Razonamientos \\
		\midrule
		- Pasar de forma binómica a forma polar \\
		- Pasar de forma binómica y polar a forma trigonométrica\\
		- Operaciones con números complejos en forma binómica: sumar, \\ \hspace{0.2cm}multiplicar y dividir números complejos. Producto por escalar y potencia. \\
		- Operaciones con números complejos en forma polar: Producto, \\
		\hspace{0.2cm}división, raíz $n$-ésima y potencia. Uso de la fórmula de Moivre.\\
		\bottomrule
	\end{tabular}
	\caption{Contenido procedimental: Razonamientos}
	\label{tab:razonamientos}
\end{table}



\begin{table}[H]
	\centering
	\begin{tabular}{lcccccc}
		\toprule
		\hspace{5.5cm}Estrategias \\
		\midrule
		\bottomrule
	\end{tabular}
	\caption{Contenido procedimental: Estrategias}
	\label{tab:estrategias}
\end{table}


\subsection{Sistemas de Representación}

\subsubsection{Representación numérica, simbólica o algebraica}

\subsubsection{Representación verbal}

\subsubsection{Representación gráfica}

\subsubsection{Recursos digitales}

\subsection{Sentido del Concepto Matemático}

\subsubsection{Términos}


\subsubsection{Contextos Matemáticos}

\subsubsection{Fenómenos}

Tal y como expresa \cite{lupi2013} en \cite{rico2013}, la historia es un buen mecanismo para estudiar el contenido de un tema, pues el desarrollo histórico es útil para determinar el origen de los conceptos, comparar sistemas de representación o localizar problemas clásicos.


\subsubsection{Situaciones}
\end{document}

%%% Local Variables:
%%% mode: latex
%%% TeX-master: "../main"
%%% End:
