\documentclass[../main.tex]{book}

\begin{document}

\chapter{Fundamentación teórica. Contexto y Aspectos Legales}


Toda unidad didáctica se basa en el conjunto de directrices y conceptos establecidos en la normativa vigente. El pilar fundamental es el currículo. De acuerdo con lo establecido en el Real Decreto 1105/2014, de 26 de diciembre, en su artículo 2, se entiende por currículo la \textit{``regulación de los elementos que determinan los procesos de enseñanza y aprendizaje para cada una de las enseñanzas y etapas educativas''}. \\

El desarrollo del currículo es complejo y se realiza a distintos niveles, llamados niveles de concreción curricular,que garantizan que el mismo tenga la suficiente riqueza y que se adapte de lo general a lo concreto, esto es, que tenga  en cuenta la diversidad  de las comunidades autónomas, los centros educativos y cada clase. Se definen tres niveles de concreción curricular:

\begin{enumerate}
	\item \textbf{Normativas.} Ley, Real Decreto, Decreto, Orden e Instrucción, recogen la legislación proveniente de la autoridad educativa competente. Destacan (en el contexto de España y la Comunidad Autónoma de Andalucía) la Ley Orgánica 3/2020, de 29 de diciembre; el Real Decreto 1105/2014, de 26 de diciembre; los Decretos 182/2020 y 111/2016; la Orden de 15 de enero de 2021 y la Orden ECD/65/2015; y la Instrucción 9/2020.
	
	\item \textbf{Plan de Centro.} Formado por el Proyecto educativo, elaborado por cada centro y que refleja su identidad, objetivos y organización; el Reglamento de Organización y Funcionamiento (ROF), donde se recogen las normas organizativas y funcionales del centro; y el Proyecto de Gestión, donde se establecen los criterios a seguir con respecto a la gestión de los recursos humanos, materiales y económicos del centro.
	
	\item \textbf{Programación de Aula.} Se entiende por Programación de Aula el ajuste del Plan de Centro para un curso o clase, delimitando los objetivos previstos para este grupo y la hoja de ruta para llevarlos a cabo.
\end{enumerate}

Ciertos autores, como ..., incluyen un cuarto nivel de concreción curricular, llamado \textbf{Nivel Adaptativo}, en el que se adapta el Plan de Centro a un alumno o a un grupo muy reducido. Habitualmente, está relacionado con la atención a la diversidad o a estudiantes con necesidades educativas especiales.






\end{document}

%%% Local Variables:
%%% mode: latex
%%% TeX-master: "../main"
%%% End: