\documentclass[../main.tex]{book}

\begin{document}

\chapter{Fundamentación teórica. Contexto y Aspectos Legales}
\label{sec:fund-teorica}

Toda unidad didáctica se basa en el conjunto de directrices y conceptos establecidos en la normativa vigente. El pilar fundamental es el currículo. De acuerdo con lo establecido en el Real Decreto 1105/2014, de 26 de diciembre, en su artículo 2, se entiende por currículo la \textit{``regulación de los elementos que determinan los procesos de enseñanza y aprendizaje para cada una de las enseñanzas y etapas educativas''}. \\

El desarrollo del currículo es complejo y se realiza a distintos niveles, llamados niveles de concreción curricular, que garantizan la riqueza necesaria y su adaptación de lo general a lo concreto, esto es, que tenga  en cuenta la diversidad  de las comunidades autónomas, los centros educativos y cada clase. Se definen tres niveles de concreción curricular:

\begin{enumerate}
	\item \textbf{Normativas.} Ley, Real Decreto, Decreto, Orden e Instrucción, recogen la legislación proveniente de la autoridad educativa competente. Destacan (en el contexto de España y la Comunidad Autónoma de Andalucía) la Ley Orgánica 3/2020, de 29 de diciembre; el Real Decreto 1105/2014, de 26 de diciembre; los Decretos 182/2020 y 111/2016; la Orden de 15 de enero de 2021 y la Orden ECD/65/2015; y la Instrucción 9/2020.
	
	\item \textbf{Plan de Centro.} Formado por el Proyecto educativo, elaborado por cada centro y que refleja su identidad, objetivos y organización; el Reglamento de Organización y Funcionamiento (ROF), donde se recogen las normas organizativas y funcionales del centro; y el Proyecto de Gestión, donde se establecen los criterios a seguir con respecto a la gestión de los recursos humanos, materiales y económicos del centro.
	
	\item \textbf{Programación de Aula.} Se entiende por Programación de Aula el ajuste del Plan de Centro para un curso o clase, delimitando los objetivos previstos para este grupo y la hoja de ruta para llevarlos a cabo.
\end{enumerate}

Ciertos autores, como \cite{cabrerizo2007}, incluyen un cuarto nivel de concreción curricular, llamado \textbf{Nivel Adaptativo}, en el que se adapta el Plan de Centro a un alumno o a un grupo muy reducido. Habitualmente, está relacionado con la atención a la diversidad o a estudiantes con necesidades especiales de apoyo educativo. \\

Nuestra unidad docente recibe influencias de todos los niveles de concreción curricular, aunque siempre se tiende a adaptar lo planificado a la situación de cada grupo y, especialmente, se debe atender a la diversidad presente en dicho grupo. Antes de presentar los contenidos, competencias, criterios de evaluación y estándares de aprendizaje referentes a la temática que nos ocupa, merece la pena darles una definición rigurosa, en aras de un mejor entendimiento, planificación y uso de los mismos. Acudimos de nuevo al artículo 2 del Real Decreto 1105/2014, de 26 de diciembre, en el que encontramos las definiciones básicas de los conceptos que se utilizarán a lo largo de esta unidad. Son los siguientes:

\begin{itemize}
	\item \textbf{Objetivos: } Referentes relativos a los logros que el estudiante debe alcanzar al finalizar cada etapa, como resultado de las experiencias de enseñanza-aprendizaje intencionadamente planificadas a tal fin.
	\item \textbf{Competencias: } capacidades para aplicar de forma integrada los contenidos propios de cada enseñanza y etapa educativa, con el fin de lograr la realización adecuada de actividades y la resolución eficaz de problemas complejos.
	\item \textbf{Contenidos: } Conjunto de conocimientos, habilidades, destrezas y actitudes que contribuyen al logro de los objetivos de cada enseñanza y etapa educativa y  a la adquisición de competencias. Los contenidos se ordenan en asignaturas, que se clasifican en materias y ámbitos, en función de las etapas educativas o los programas en que participe el alumnado.
	\item \textbf{Estándares de aprendizaje evaluables: } especificaciones de los criterios de evaluación que permiten definir los resultados de aprendizaje, y que concretan lo que el estudiante de be saber, comprender y saber hacer en cada asignatura; deben ser observables, medibles y evaluables y permitir graduar el rendimiento o logro alcanzado. Su diseño debe contribuir y facilitar el diseño de pruebas estandarizadas y comparables.
	\item \textbf{Criterios de evaluación: } son el referente específico para evaluar el aprendizaje del alumnado. Describen aquello que se quiere valorar y el que alumnado debe lograr, tanto en conocimientos como en competencias; responden a lo que se pretende conseguir en cada asignatura.
	\item \textbf{Metodología didáctica: conjunto de estrategias, procedimientos y acciones organizadas y planificadas por el profesorado, de manera consciente y reflexiva, con la finalidad de posibilitar el aprendizaje del alumnado y el logro de los objetivos planteados.}
\end{itemize} 

Una vez fijados todos los conceptos clave en el desarrollo de cualquier unidad didáctica, podemos focalizarnos en la que nos ocupa. Como ya comentamos en la sección anterior, esta unidad didáctica está diseñada para los alumnos de la asignatura Matemáticas I, de 1º de Bachillerato. En particular, de conformidad con la Orden del 14 de julio de 2016, por la que se desarrolla el currículo correspondiente al Bachillerato en  la  Comunidad  Autónoma  de  Andalucía,  se  regulan  determinados  aspectos  de  la  atención  a  la diversidad y se establece la ordenación de la evaluación del proceso de aprendizaje del alumnado; el Decreto 110/2016, de 14 de junio; y con el Real Decreto 1105/2014, de 26 de diciembre, nos centramos en los contenidos referentes a Números Complejos del Bloque 2, Números y Álgebra. Concretamente, a los siguientes contenidos, criterios y estándares: \\


\underline{Contenidos:}

\begin{itemize}
	\item Números complejos. Forma binómica y polar. Representaciones gráficas. Operaciones elementales. Fórmula de Moivre.
\end{itemize}

\underline{Criterios de evaluación:}

\begin{enumerate}
	\item Conocer los números complejos como extensión de los números reales, utilizándolos para obtener soluciones de algunas ecuaciones algebraicas (CMCT, CAA).
\end{enumerate}

siendo CMCT Competencia matemática y competencias básicas en ciencias y tecnología, y CAA Competencia Aprender a aprender, competencias clave recogidas en la Ley Orgánica 8/2013, de 9 de diciembre, para la mejora de la calidad educativa. Por último, \\

\underline{Estándares de aprendizaje evaluables:}

\begin{enumerate}
	\item Valora los números complejos como ampliación del concepto de números reales y los utiliza para obtener la solución de ecuaciones de segundo grado con coeficientes reales sin solución real.
	\item Opera con números complejos, los representa gráficamente y utiliza la fórmula de Moivre en el caso de las potencias.
\end{enumerate}


Estos contenidos son especialmente importantes por varios motivos. En primer lugar, porque es la primera (y última) vez que el alumnado se enfrenta a los números complejos en su etapa de Educación Secundaria Obligatoria y Bachillerato. Además, desde que empiezan a estudiar ecuaciones de segundo grado, se encuentran con que algunas de ellas, aparentemente sencillas, no tienen solución. Con este bloque se cierra ese círculo, resolviéndose incertidumbres arrastradas durante años. Depositaremos todo nuestro rigor en las siguientes secciones para diseñar una estrategia de aprendizaje eficiente, que ligue la visión geométrica y algebraica de los números complejos, para conseguir que los alumnos y alumnas acaben el curso con una percepción amplia y rica de todos los conjuntos numéricos, incluyendo el cuerpo de los números complejos, y todas sus aplicaciones.


\end{document}

%%% Local Variables:
%%% mode: latex
%%% TeX-master: "../main"
%%% End: